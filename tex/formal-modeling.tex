Three elements of of narrative-based systems enable robust
human-machine interaction: First, there exist good techniques for
representing narratives, including automata-based representations for
encoding operator intent.  Second, a narrative can be represented as a
trajectory through a state space of possible situations.  A mission is
a planned trajectory through a state space, with agent choices and
environment input alternative trajectories. Third, narrative-based
systems support flexible allocations of authority and autonomy, thus
allowing distributed humans and machines to robustly perform a
mission.

We propose two classes of models.  First, we propose to use
(deterministic) timed automata to explicitly represent the set of
possible behaviors of agents in the system.  This builds on our
previous work (Modeling UASs for Role Fusion and Human Machine
Interface Optimization; Gledhill, Mercer, \& Goodrich, Proc of IEEE
Conf on Sys, Man, and Cybernetics, 2013).  These timed automata are
appropriate for human-machine interaction because they implicitly
represent the set of afforded behaviors of the team.

Second, we propose to use Markov chains to represent likely outcomes
of agent behaviors.  Systems that operate in the real environment with
real humans must be robust to deviations from a deterministic plan.
Probabilistic models allow us to quantify performance bounds as a
function of level of uncertainty and to inform a human operator of the
level of persistence required to produce a robust outcome (Abstraction
and Persistence: Macro-Level Guarantees of Collective Bio-Inspired
Teams under Human Supervision; Goodrich \& Mercer, Proc of
Infotech@Aerospace, 2012).

Model checking is particularly effective in isolating violations of
system-level properties. A user posing "what-if" scenarios is able to
assess possible outcomes and quickly isolate trajectories that enter
high-risk, high workload, or failure situations. The proposed
verification approaches will (a) use symbolic execution and SMT
technology to manage state explosion and (b) leverage advances in
probabilistic analysis to express bounds on high-risk
trajectories. This yields the ability to detect and predict problems,
with guidance on how to mediate these problems.

\textbf{Eric, I think that the key elements for your writing derive from the following excerpts from the call:}

The focus should be on human-human as well as human-automation
interactions and the exploration of off-nominal scenarios that can
emerge from these interactions. It is important to take into account
the stochastic elements of the NAS. For example, the analyses of
interest could include the following, non-exhaustive list of
off-nominal scenarios:
\begin{itemize}
\item Misunderstandings due to mixed equipage
\item Sudden deviations from planned trajectories due to exceptional circumstances 
such as unexpected pilot actions or unexpected critical hardware or software 
failures
\item Safety issues resulting from overly competitive behaviors or from gamesmanship.
\end{itemize}

In particular this solicitation seeks techniques that can handle stochastic events and 
emerging behaviors. The techniques of interest include:
\begin{itemize}
\item Game theory: in particular for modeling human aspects involved in decision 
making between humans, but in the context of interaction with some automation.
\item Bayesian analysis: for modeling probability propagation throughout a graph of 
modeling interactions between human and automated agents.
\item Learning and coordination in multi-agent systems: for capturing the evolution of 
human behaviors over time and especially the learning of specific situations and 
the optimization or simplification of procedures in air traffic that can lead to 
confusion for other agents
\end{itemize}

The idea would be to state what can and can’t be done using existing
techniques in model-checking, leading to the idea of extramodel
information and the need for constrained negotiation.  State what is
unique about our approach.

An Unmanned Aircraft Systems (UAS) moving along the periphery of an
airport suddenly deviates into the controlled airspace.  This might be
due to human error, automation error, or system failure. The ATC
informs the UAS it is in violation of the airspace and directs it to
resume nominal flight path. Concurrently the ATC must assess the
immediate impact of this event on all aircraft in the area near the
UAS. This includes such impacts as potential separation violations and
delays that could adversely reduce fuel margins. If the UAS cannot
resume its nominal flight path, the ATC must now recommend mitigating
actions to the UAS. These recommendations must take into account any
loss of functionality sustained by the UAS. The impact of these
mitigating actions must be assessed on both the UAS and the other
aircraft in the area. These impacts must be projected forward in time
and through the airspace. The ATC also must work with other ATCs to
assess the impact of the anomalous event and mitigating actions on the
aircraft managed by them. Tools and techniques are needed to help ATCs
understand and predict UAS behaviors and their impacts on each other
and on crewed aircraft. These techniques should support the evaluation
and comparison of alternative actions taken to mitigate adverse
impacts.




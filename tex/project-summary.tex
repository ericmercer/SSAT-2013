%%%%%%%%%%%%%%%%%%%%%%%%%%%%%%%%%%%%%%%%%%%%%%%%%%%%%%%%%%%%%%%%%%%%

\documentclass[palatinofont10]{nsfprop}

\thispagestyle{empty}
\pagenumbering{arabic}
\begin{document}

\begin{center}
\textbf{Combining Narrative-Based Representations and Model-Checking for Robust Supervised Autonomy}
\end{center}

The concept of narrative applies to human-machine systems because it
(a) enables an explicit representation for interactions between
actors, (b) is amenable to formal modeling, and (c ) is compatible
with summaries that enable human understanding. The explicit
representation of narrative in emerging systems produces interactive
systems that are engaging to humans and compatible with the ways
humans understand and predict complex behaviors. Moreover, a narrative
framework is a natural way to consider how the activities of one agent
constrain or influence the activities of another, allowing a formalism
for analyzing how different roles affect a mission.

Three elements of narrative-based systems enable robust human-machine
interaction: First, there exist good techniques for representing
narratives, including automata-based representations for encoding
operator intent. Second, a narrative can be represented as a
trajectory through a state space of possible situations. A mission is
a planned trajectory through a state space, with agent choices and
environment input alternative trajectories. Third, narrative-based
systems support flexible allocations of authority and autonomy, thus
allowing distributed humans and machines to robustly perform a
mission.

We propose two classes of models. First, we propose to use
(deterministic) timed automata to explicitly represent the set of
possible behaviors of agents in the system. This builds on our
previous work (Modeling UASs for Role Fusion and Human Machine
Interface Optimization; Proc of IEEE Conf on Sys, Man, and
Cybernetics, 2013). These timed automata are appropriate for
human-machine interaction because they implicitly represent the set of
afforded behaviors of the team.

Second, we propose to use Markov chains to represent likely outcomes
of agent behaviors. Systems that operate in the real environment with
real humans must be robust to deviations from a deterministic
plan. Probabilistic models allow us to quantify performance bounds as
a function of level of uncertainty and to inform a human operator of
the level of persistence required to produce a robust outcome
(Abstraction and Persistence: Macro-Level Guarantees of Collective
Bio-Inspired Teams under Human Supervision; Proc of
Infotech@Aerospace, 2012).

Model checking is particularly effective in isolating violations of
system-level properties. A user posing "what-if" scenarios is able to
assess possible outcomes and quickly isolate trajectories that enter
high-risk, high workload, or failure situations. The proposed
verification approaches will (a) use symbolic execution and SMT
technology to manage state explosion and (b) leverage advances in
probabilistic analysis to express bounds on high-risk
trajectories. This yields the ability to detect and predict problems,
with guidance on how to mediate these problems.

Narrative-based representations provide a comprehensible and flexible
way to explore agent behaviors and interactions with their environment
over time and from different perspectives of inquiry. We propose to
use narrative-based performance summaries, building on work by the
proposers for remote supervision of NASA planetary robots (NASA Phase
II STTR. Anytime Summarization for Remote Robot Operations). These
summaries orient personnel quickly about the performance of remote
agents performing complex tasks with variable levels of
autonomy. Summary measures identify what progress the agents have made
and, when progress is impeded, indicate what went wrong. Trending
measures determine how well agent assets are being utilized, identify
opportunities to improve agent productivity, and detect impacts to
productivity resulting from degraded agent capabilities. Key to our
approach is the ability to summarize important differences between
actual performance and performance expectations.
\paragraph{Keywords:\/} ???ADD KEYWORDS???

\end{document}

The explicit representation of narrative in emerging systems
(http://www.cc.gatech.edu/~riedl/pubs/aimag.pdf) produce interactive
systems that are engaging to humans and compatible with the ways
humans understand and predict complex behaviors.  Moreover, a
narrative framework is a natural way to consider how the activities of
one agent constrain or influence the activities of another, allowing a
formalism for analyzing how different roles affect a mission.
Importantly, these explicit representations are compatible with a
suite of algorithms from computer science.

\textbf{Mike: Here’s what is said in the project description.}

“At this time, NASA is specifically seeking techniques that can handle
the stochastic and complex nature of many elements (including
automated systems, humans, environment conditions) in the NAS,
especially the unpredictable nature of human responses and the safety
aspects involved in trying to “game the system”. … NASA is seeking
analysis techniques that complement traditional human factor studies
by addressing off-nominal conditions and coping with variability.”

Discuss how narrative is a useful metaphor for describing things that
are inherently based in a network.  Swarming and social
analytics-based techniques to the NAS problem assume scales that are
too large for the types of networks involved, and deterministic
centralized controllers assume scales that are too small.  Pose the
problem of designing and managing a networked team with information
passing among agents.  State that the objective is to produce dynamics
that induce acceptable attractors in the system, where the attractors
are an efficient distribution over space and time of vehicles and
operator workload.

Discuss how individuals in the distributed network will not be able to
(nor interested in) knowing everything about the entire network, but
rather need an appropriate level of abstraction for things happening
beyond their immediate neighbors in the network.  Narrative is the way
that we create this abstraction, including forming a probabilistic
mapping between a human’s or autonomous agent’s actions and
consequences for other actors in the narrative.

Describe what can be done with deterministic approaches and how
non-deterministic approaches allow us more flexibility in narrative
abstraction provided that we know how to act within these narratives.
Then, introduce the principles of persistence as a means for exerting
appropriate influence for important needs, and negotiation as a means
for reaching Pareto efficient solutions given the rules of networked
interactions.

Present the model as one with underlying game theory principles
encoded in as a mechanism design problem, state the limits of
algorithmic approaches to solving this problem, and then introduce the
need for appropriate (narrative) metaphors that will allow humans to
form accurate and useful mental models for making choices within the
system.  These same metaphors will serve as the basis for helping
autonomy to know what (and at what level of abstraction) to
communicate with humans in the system.

